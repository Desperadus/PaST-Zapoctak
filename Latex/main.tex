\documentclass[12pt]{article}
\usepackage[superscript,biblabel]{cite}
\usepackage{blindtext}
\usepackage{hyperref}

\usepackage[]{fullpage}
\usepackage{pythonhighlight}


\title{Míra sebevražd ve světě}
\author{Tomáš Jelínek}
\date{Červenec 2023}

\hypersetup{
    colorlinks=true,
    linkcolor=blue,
    filecolor=magenta,      
    urlcolor=blue,
    pdfpagemode=FullScreen,
    }

\begin{document}
\maketitle
\abstract{Míra sebevražd je dobrým indikátorem duševního zdraví populace, jelikož bývají vyvrcholením závažných psychických \cite{bertolote2002global, bertolote2004psychiatric, chang2011depressed, ferrari2014burden} problémů jedince. Nalezení jevů asociovaných se sebevraždami a zkoumání efektivity nástrojů s cílem snížit jejich míru je esenciální proto, abychom jim co nejvíce předcházeli. V této statistické práci se zaměřím jak na analýzu rizik, tak na efektivitu duševní zdravotní péče.}\\

Data jsou brána z volně dostupného datasetu na kaggle \href{https://www.kaggle.com/datasets/twinkle0705/mental-health-and-suicide-rates?select=Facilities.csv}{Mental Health and Suicide Rates}.  

\section{Sebevraždy u mužů a žen}
\begin{python}
print pokus
\end{python}

\bibliographystyle{abbrv}
\bibliography{citations}
\end{document}
