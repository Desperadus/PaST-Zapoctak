\documentclass[12pt]{article}
\usepackage[superscript,biblabel]{cite}
\usepackage{blindtext}
\usepackage{hyperref}
\usepackage{pdfpages}

\usepackage[]{fullpage}
\usepackage{pythonhighlight}
\usepackage{graphicx}
\usepackage{caption}


\title{Míra sebevražd ve světě}
\author{Tomáš Jelínek}
\date{Červenec 2023}

\hypersetup{
    colorlinks=true,
    linkcolor=blue,
    filecolor=magenta,      
    urlcolor=blue,
    pdfpagemode=FullScreen,
    }

\begin{document}
\maketitle
\abstract{Míra sebevražd je dobrým indikátorem duševního zdraví populace, jelikož bývají vyvrcholením závažných psychických \cite{bertolote2002global, bertolote2004psychiatric, chang2011depressed, ferrari2014burden} problémů jedince. Nalezení jevů asociovaných se sebevraždami a zkoumání efektivity nástrojů s cílem snížit jejich míru je esenciální proto, abychom jim co nejvíce předcházeli. V této statistické práci se zaměřím jak na analýzu rizik, tak na efektivitu duševní zdravotní péče.}\\

Data jsou brána z volně dostupného datasetu na kaggle \href{https://www.kaggle.com/datasets/twinkle0705/mental-health-and-suicide-rates?select=Facilities.csv}{Mental Health and Suicide Rates}.  

\section{Sebevraždy u mužů a žen}
Podívejme se nyní, jestli jsou obě dvě skupiny obyvatel pod stejným rizikem. Buď $H_0$, že ženy a muži jsou ve světě pod stejným rizikem a buď $H_1$ že jedno pohlaví páchá více sebevražd. Hypotézu ověřím dvouvýběrovým t-testem a jako hladinu významnosti volím $\alpha = 0.05$ . Ten má určité předpoklady.
\begin{enumerate}
\item Nezávislost jevů - lze předpokládat, že drtivá většina sebevražd bude nezávislá na jiných.
\item Normální rozdělení - lze očekávat rozdělení podobné normálnímu, ovšem ne přímo normální. (Extrémy a míra sebevražd nemůže být záporná nám v tom brání). Ovšem dataset je docela velký $(n > 30)$ a tedy díky centrální limitní větě si můžeme dovolit udělat t-test.
\item Poslední podmínkou k užití je to, že variance budou hodně podobné - vzhledem k tomu, že netuším, zdalipak budou podobné, tak zvolím Welchův t-test, který tohle nepředpokládá a umožňuje je mít rozdílné.
\end{enumerate}

\newpage
Zde je analýza oněch dat v pythonu.
\begin{python}
def welch_ttest():
	df = pd.read_csv("data/Age-standardized suicide rates.csv")
    df_male = df.loc[df["Sex"].str.contains("Male")]
    df_female = df.loc[df["Sex"].str.contains("Female")]
    np_males df_males["2016"].to_numpy()
    np_females = df_females["2016"].to_numpy()
    t_statistic, p_value = stats.ttest_ind(np_males, np_females, equal_var = False)
    return(t_statistic, p_value)
\end{python}
\paragraph{}
Jako výstup dostáváme: $t\_statistic = 13.4164773450928$ a $p\_value = 2.8621990375255226^{-31}$. Vzhledem k tomu, že $p\_value$ je výrazně nížší než naše $\alpha$, tak $H_0$ zamítáme. Pro zajímavost zde je barplot průměrné míry sebevražd pro obě pohlaví. 
\begin{figure}[hbt]
  \centering
  \includegraphics[width=0.8\textwidth]{../Coding/genders_mean_suicide_rate.pdf}
  %\caption{Caption of the plot}
  \label{fig:plot}
\end{figure}

\newpage
A tabulka zemí, kde byla míra sebevražd žen v roce 2016 vyšší. Ve zbylých 176 zemích převažovala u mužů.
\begin{table}[htb]
\centering
%\caption{Míra sebevražd v zemích, kde se zabilo více žen}
\label{tab:suicide-rates}
\begin{tabular}{lccc}
\hline
Country & Female & Male \\
\hline
Antigua and Barbuda & 0.9 & 0.0 \\
Bangladesh & 6.7 & 5.5 \\
China & 8.3 & 7.9 \\
Lesotho & 32.6 & 22.7 \\
Morocco & 3.6 & 2.5 \\
Myanmar & 9.8 & 6.3 \\
Pakistan & 3.1 & 3.0 \\
\hline
\end{tabular}
\end{table}

\section{Štěstí, GDP a míra sebevražd}


\renewcommand\refname{Reference}
\bibliographystyle{abbrv}
\bibliography{citations}
\end{document}
