\documentclass[12pt]{article}
\usepackage[superscript,biblabel]{cite}
\usepackage{blindtext}
\usepackage{hyperref}
\usepackage{pdfpages}

\usepackage[]{fullpage}
\usepackage{pythonhighlight}
\usepackage{graphicx}
\usepackage{caption}
\usepackage[utf8x]{inputenc}
\usepackage[czech,english]{babel}


\title{Míra sebevražd ve světě}
\author{Tomáš Jelínek}
\date{Červenec 2023}

\hypersetup{
    colorlinks=true,
    linkcolor=blue,
    filecolor=magenta,      
    urlcolor=blue,
    pdfpagemode=FullScreen,
    }

\begin{document}
\maketitle
\abstract{Míra sebevražd je jedním z možných indikátorů duševního zdraví populace, jelikož bývají vyvrcholením závažných psychických \cite{bertolote2002global, bertolote2004psychiatric, chang2011depressed, ferrari2014burden} problémů jedince. Navíc jsou jasně měřitelným ukazatelem oproti prevalenci duševních poruch v populaci. V této statistické práci se zaměřím na analýzu jejich míry dle pohlaví a jak jsou ovliňovány různými faktory jako je dostupnost duševní péče v dané zemi, GDP nebo štěstí populace.}\\

Data jsou brána z volně dostupného datasetu na kaggle \href{https://www.kaggle.com/datasets/twinkle0705/mental-health-and-suicide-rates?select=Facilities.csv}{Mental Health and Suicide Rates}. Ten vychazí z dat WHO - moje analýzy jsou dělány na údajích z roku 2016. Zdrojové kódy jsou volně dostupné na mém \href{https://github.com/Desperadus/PaST-Zapoctak}{GitHubu}.

\section{Sebevraždy u mužů a žen}
Otázkou je, zda obě dvě skupiny obyvatel jsou pod stejným rizikem. Buď $H_0$, že ženy a muži jsou ve světě pod stejným rizikem a buď $H_1$ že jedno pohlaví má větší šanci na sebevraždu. Hypotézu ověřím dvouvýběrovým t-testem a jako hladinu významnosti volím $\alpha = 0.05$ . T-test má určité předpoklady.
\begin{enumerate}
\item Nezávislost jevů - lze předpokládat, že drtivá většina sebevražd bude nezávislá na jiných.
\item Normální rozdělení - dataset je velký $(n > 30)$ a tedy díky centrální limitní větě si můži dovolit udělat t-test, i když data nemusí sama nutně mít normální rozdělení.
\item Poslední podmínkou k užití je to, že variance budou hodně podobné - vzhledem k tomu, že netuším jaké jsou, tak zvolím Welchův t-test, který tohle nepředpokládá a umožňuje je mít rozdílné.
\end{enumerate}

\newpage
Zde je analýza oněch dat v pythonu.
\begin{python}
import numpy as np
import pandas as pd
import scipy.stats as stats
def welch_ttest():
		df = pd.read_csv("data/Age-standardized suicide rates.csv")
    df_male = df.loc[df["Sex"].str.contains("Male")]
    df_female = df.loc[df["Sex"].str.contains("Female")]
    np_males df_males["2016"].to_numpy()
    np_females = df_females["2016"].to_numpy()
    t_statistic, p_value = stats.ttest_ind(np_males, np_females, equal_var = False)
    return(t_statistic, p_value)
print(welch_ttest())
\end{python}
\paragraph{}
Jako výstup dostávám: $t\_statistic = 13.4164773450928$ a $p\_value = 2.8621990375255226^{-31}$. Vzhledem k tomu, že $p\_value$ je výrazně nížší než na začátku zvolená $\alpha$, tak $H_0$ zamítám. Pro zajímavost zde je barplot průměrné míry sebevražd pro obě pohlaví. 
\begin{figure}[hbt]
  \centering
  \includegraphics[width=0.8\textwidth]{../Coding/genders_mean_suicide_rate.pdf}
  %\caption{Caption of the plot}
  \label{fig:plot}
\end{figure}

\newpage
A tabulka zemí, kde byla míra sebevražd žen v roce 2016 vyšší než u mužů. Ve zbylých 176 zemích převažovala u mužů.
\begin{table}[htb]
\centering
%\caption{Míra sebevražd v zemích, kde se zabilo více žen}
\label{tab:suicide-rates}
\begin{tabular}{lccc}
\hline
Country & Female & Male \\
\hline
Antigua and Barbuda & 0.9 & 0.0 \\
Bangladesh & 6.7 & 5.5 \\
China & 8.3 & 7.9 \\
Lesotho & 32.6 & 22.7 \\
Morocco & 3.6 & 2.5 \\
Myanmar & 9.8 & 6.3 \\
Pakistan & 3.1 & 3.0 \\
\hline
\end{tabular}
\end{table}

\newpage
\section{Léčebné instituce, štěstí a míra sebevražd}
\subsection{Léčebné instituce}
	Mezi nástroje pro zlepšovaní mentálního zdraví patří odborná péče od psychiatrů, psychologů, sociálních pracovníků atd.\ Bude tedy míra sebevražd lepší v zemích, kde je péče dostupnější? Na to vyrobím korelační matici: \\

\begin{python}
#Other imports like in previous code snippet
import seaborn as sns
df = load_merged_data() #more about this on my GitHub
df = df.loc[df["Sex"].str.contains("Both sexes")]
df = df.rename(columns={"2016": "suicide_rate"})
df = df.loc[:, ["suicide_rate", "Psychiatrists","Psychologists" ,"Nurses", "Social_workers", "Happiness Rank", "Mental _hospitals", "outpatient _facilities", "day _treatment", "residential_facilities"]]
df.to_csv("correlation_matrix.csv")
corr_matrix = df.corr()
fig, ax = plt.subplots(figsize=(8, 6))
sns.heatmap(corr_matrix, annot=True, cmap="coolwarm", ax=ax)
plt.show()
\end{python}

\begin{figure}[hbt]
  \centering
  \includegraphics[width=0.65\textwidth]{../Coding/corr_matrix.pdf}
  %\caption{Caption of the plot}
  \label{fig:plot}
\end{figure}

Počty psychiatrů/psychologů, mentálních institucí atd. na hlavu nemají samy o sobě téměř žádnou korelaci s mírou sebevražd, i když by člověk na první pohled asi očekával, že budou mít negativní. 
\subsection{Štěstí}
Zajímavé je, že ani rank státu v žebříčku šťastnosti obyvatel s tím nekoreluje, byť je v datech patrná mírna polynomiální zavislost. (Data štěstí jsou z tohoto \href{https://www.kaggle.com/datasets/unsdsn/world-happiness}{datasetu}). 

\begin{figure}[hbt]
  \centering
  \includegraphics[width=0.57\textwidth]{../Coding/polynomial_regression_happiness.pdf}
  %\caption{Sebevraždy a štěstí}
  \label{fig:plot}
\end{figure}

\begin{python}
def happiness_regression(df, degree):
    df = df.loc[df["Sex"].str.contains("Both sexes")]

    x = df["Happiness Rank"]
    y = df["2016"]

    coeffs = np.polyfit(x, y, degree)
    poly_eqn = np.poly1d(coeffs)

    plt.scatter(x, y)
    plt.plot(x, poly_eqn(x), color="red")

    plt.xlabel("Happiness rank")
    plt.ylabel("Pocet sebevrazd na 100k obyvatel")

    plt.show()
\end{python}

\newpage
\subsection{GDP} Člověk někdy na \href{https://www.youtube.com/watch?v=brEU5j4IsxU}{internetu narazí na tvrzení}, že v bohatších zemích lidé častěji páchají sebevraždy.
\begin{figure}[hbt]
  \centering
  \includegraphics[width=0.75\textwidth]{../Coding/gdp_and_rates.pdf}
  %\caption{Sebevraždy a štěstí}
  \label{fig:plot}
\end{figure}

\begin{python}
def gdp_rates_plot(df, region = None):
    df = df.loc[df["Sex"].str.contains("Both sexes")]
    if region is not None:
        df = df.loc[df["Region"].str.contains(region)]
    df = df.dropna(subset=['2016_x', '2016_y'])
  
    np_rates = df["2016_x"].to_numpy()
    np_gdp = df["2016_y"].to_numpy()

    plt.scatter(np_gdp, np_rates)
    plt.xlabel("Gdp na hlavu")
    plt.ylabel("Pocet sebevrazd na 100k obyvatel")

    plt.show()
\end{python}

Z plotu dat je vidět, že u tohoto tvrzení musí být člověk opatrný. Byť je v bohatších zemích míra sebevražd značně vyšší než v některých chudších, existují i chudé oblasti, kde je míra sebevražd vyšší nebo stejná jak v bohatých regionech.

\section{Slovo závěrem}
Pokud vás případně zajímají další statistiky, jako je například vývoj sebevražd v čase, jiné regrese, či analýza dle regionů (Evropa, Afrika...). Můžete si je nechat udělat ze zdrojáků na mém \href{https://github.com/Desperadus/PaST-Zapoctak}{GitHubu}.

\renewcommand\refname{Reference} 
\bibliographystyle{abbrv}
\bibliography{citations}
\end{document}
